\documentclass{article}
\usepackage{graphicx} % Required for inserting images
\usepackage{amsmath}
\usepackage{float}
\usepackage{natbib}

%\bibliographystyle{plain}
\bibliographystyle{abbrvnat}

\title{RR project}
\author{Palina Tkachova}
\date{May 2024}

\begin{document}

\maketitle

\section{Introduction}

This is the introduction section.

\subsection{Subsection}

This is a subsection within the introduction.

\subsection{Mathematical Representations}

When exploring complex mathematical concepts, LaTeX provides versatile environments for displaying equations. You can use environments like array for matrices, cases for piecewise functions, and custom commands for recurring structures.

Matrix Representation using array Environment

\[
\begin{array}{ccc}
1 & 0 & 0 \\
0 & 1 & 0 \\
0 & 0 & 1 \\
\end{array}
\]

Piecewise Function using cases Environment

\[
f(x) = 
\begin{cases} 
x^2 & \text{if } x \geq 0 \\
-x & \text{if } x < 0 
\end{cases}
\]

Custom Command for Recurring Structures

\newcommand{\diff}[2]{\frac{d #1}{d #2}}
\[
\diff{y}{x} = 3x^2
\]

Exponential and Logarithmic Functions with align Environment

\begin{align}
    e^x &= 1 + x + \frac{x^2}{2!} + \frac{x^3}{3!} + \cdots \\
    \log(1+x) &= x - \frac{x^2}{2} + \frac{x^3}{3} - \frac{x^4}{4} + \cdots
\end{align}

\subsection{Lists}

\begin{itemize}
  \item First item
  \item Second item
    \begin{itemize}
      \item Sub-item 1
      \item Sub-item 2
    \end{itemize}
  \item Third item
     \begin{enumerate}
      \item Sub-item A
      \item Sub-item B
    \end{enumerate}
    
\end{itemize}

\subsection{Figures and tables}

\begin{figure}[H]
  \includegraphics[width=\linewidth]{overleaf_wide_colour_light_bg.png}
  \caption{An example image}
\end{figure}

\begin{table}[H]
  \centering
  \begin{tabular}{|c|c|}
    \hline
    Item & Quantity \\
    \hline
    Apples & 3 \\
    Oranges & 5 \\
    \hline
  \end{tabular}
  \caption{An example table}
\end{table}

And now an example of a citation \cite{goossens1994latex} in text.

\bibliography{references.bib}

\end{document}
